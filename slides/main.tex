%Use one of the two documentclass lines depending on aspect ratio needed
% for 4x3 aspect ratio slides
%\documentclass{beamer}
%for 16x9 (modern wide screen) aspect ratio slides
\documentclass[aspectratio=169,10pt]{beamer}

% Oxford Maths theming
\usetheme{oxfordmaths}
% Packages
\usepackage{amsmath,amssymb,amsfonts,amscd}
\usepackage{booktabs} %better lines in tabular
\usepackage{tcolorbox}
\usepackage{enumerate,enumitem}
\usepackage{nicematrix}
\usepackage{tikz,graphicx}
\usepackage{tikzit}
%Use one of the two documentclass lines depending on aspect ratio needed
% for 4x3 aspect ratio slides
%\documentclass{beamer}
%for 16x9 (modern wide screen) aspect ratio slides
\documentclass[aspectratio=169,10pt]{beamer}

% Oxford Maths theming
\usetheme{oxfordmaths}
% Packages
\usepackage{amsmath,amssymb,amsfonts,amscd}
\usepackage{booktabs} %better lines in tabular
\usepackage{tcolorbox}
\usepackage{enumerate,enumitem}
\usepackage{nicematrix}
\usepackage{tikz,graphicx}
\usepackage{tikzit}
\input{main.tikzstyles}
\usetikzlibrary{shapes,tikzmark,matrix}
\tikzset{every tikzmarknode/.style={%
        draw=red, semithick, inner sep=2pt}
        }
\usetikzlibrary{positioning, arrows.meta}
\usetikzlibrary{3d,perspective}
\tikzset
{
  axis/.style={thick,-latex},
  my view/.style={3d view={65}{20}},
  nutation/.style={rotate around x=\nut}
}
\usepackage{pgfplots}
\pgfplotsset{compat = newest}
\usetikzlibrary{intersections, pgfplots.fillbetween}
\ifpdf
  \DeclareGraphicsExtensions{.eps,.pdf,.png,.jpg}
\else
  \DeclareGraphicsExtensions{.eps}
\fi

\usepackage{fontawesome5}
\newcommand{\myframesubtitle}[1]{\framesubtitle{{\tiny \faChevronRight} #1}}


\newenvironment{refbox}
    { \begin{tcolorbox}[colframe=base1,colback=base3] 
        \begin{columns}
            \begin{column}[t]{0.01\textwidth}
                \centering
                \vspace{-0.4cm}
                { \faBookmark}
            \end{column}
            \begin{column}[t]{0.99\textwidth}
                \hspace{-0.4cm}
                \footnotesize}
                {
            \end{column}
        \end{columns}
    \end{tcolorbox}
    }
\usepackage{wasysym}


%%%%%%%%%%%%%%%%%%%%%%%%%%%%%%%%% MACROS %%%%%%%%%%%%%%%%%%%%%%%%%%%%%%%
\usepackage{mathtools}
\mathtoolsset{showonlyrefs}
\usepackage{todonotes}
\makeatletter
\newcommand{\mat}[1]{{\mathpalette\mat@{#1}}}
\newcommand{\mat@}[2]{%
  \begingroup
  \sbox\z@{$\m@th#1\underline{#2}$}%
  \dimen@=\dp\z@ \advance\dimen@ -2\mat@dimen{#1}%
  \dp\z@=\dimen@
  \sbox\z@{$\m@th\underline{\box\z@}$}%
  \box\z@
  \endgroup
}
\newcommand\mat@dimen[1]{%
  \fontdimen8
  \ifx#1\displaystyle\textfont\else
  \ifx#1\textstyle\textfont\else
  \ifx#1\scriptstyle\scriptfont\else
  \scriptscriptfont\fi\fi\fi 3
}
\newcommand{\tensor}[1]{{\mathpalette\tensor@{#1}}}
\newcommand{\tensor@}[2]{%
  \begingroup
  \sbox\z@{$\m@th#1\underline{#2}$}%
  \dimen@=\dp\z@ \advance\dimen@ -2\tensor@dimen{#1}%
  \dp\z@=\dimen@
  \sbox\z@{$\m@th\underline{\box\z@}$}%
  \box\z@
  \endgroup
}
\newcommand\tensor@dimen[1]{%
  \fontdimen8
  \ifx#1\displaystyle\textfont\else
  \ifx#1\textstyle\textfont\else
  \ifx#1\scriptstyle\scriptfont\else
  \scriptscriptfont\fi\fi\fi 3
}
\def\contraction{\vbox{\baselineskip2.5\p@ \lineskiplimit\z@
  \kern\p@\hbox{.}\hbox{.}\hbox{.}\hbox{.}}}
\makeatother
\newcommand{\duality}[4]{\prescript{}{#1}{\Big\langle} #2 , #3 \Big\rangle_{#4}}
\newcommand{\interp}{\newpage}
\newcommand{\abs}[1]{\lvert#1\rvert}
\newcommand{\norm}[1]{\lVert #1\rVert}
\newcommand{\cchevrons}[1]{\langle\!\langle #1 \rangle\!\rangle}
\newcommand{\ohat}[1]{\overset{\circ}{#1}}
\renewcommand{\v}{\vec{\text{v}}}
\DeclareMathOperator{\tr}{tr}
\let \vec \underline
\let \pt \mathbf

\usepackage{stmaryrd} 

% PRESENTATION INFO

\author[U. Zerbinati]{Umberto Zerbinati*} %ADD CO-AUTHORS IF NEEDED
\institute{*Mathematical Institute -- University of Oxford}

\title[Embedded Trefftz Maxwell]{
Embedded Trefftz Discretisations for the Maxwell
Eigenvalue Problem with Applications to MHD}
\date[Heidelberg, 1st Sep. '25]{ENUMATH, 1st September 2025} %[short for footer]{INSERT CONFERENCE NAME, full date for titlepage}


\begin{document}
% Setting colors for tcolorbox 
\tcbset{colframe=base02,colback=base2,coltext=base03,coltitle=base3}

\begin{frame}[plain,noframenumbering]
  \titlepage
\end{frame}

\begin{frame}
    \frametitle{Trefftz Methods}
    $\newline$
    The idea behind DG-Trefftz methods is to consider a discontinuous Galerkin method where the local approximation spaces are made of functions that are piecewise solutions of the target PDE.
    For example, let us consider the Laplace equation,
    \begin{equation}
        -\Delta u = 0 \quad \text{in } \Omega,  \quad u = g \quad \text{on } \partial \Omega.
    \end{equation} 
    \visible<2->{
        A DG-Trefftz method for this problem would consider a mesh $\mathcal{T}_h$ of $\Omega$ and a local discrete space
        \begin{equation}
            \mathbb{T}^p(K) = \{ v \in \mathbb{P}^p(K) : \Delta v = 0 \text{ in } K \}, \quad \forall K \in \mathcal{T}_h,
        \end{equation}
        where $\mathbb{P}^p(K)$ is the space of polynomials of degree at most $p$ on the element $K$.
    }
    \visible<3->{
        The global discrete space is then defined as
        \begin{equation}
            \mathbb{T}_h = \{ v_h \in L^2(\Omega) : v_h|_K \in \mathbb{T}^p(K), \forall K \in \mathcal{T}_h \}. 
        \end{equation}
    }
\end{frame}
\begin{frame}
    \textbf{No conformity} is imposed across element interfaces in the space $\mathbb{T}_h$, hence a DG formulation is needed to enforce the continuity of the solution.
    We thus consider the following DG formulation: find $u_h \in \mathbb{T}_h$ such that
    \begin{align}
        \int_{\mathcal{T}_h} \nabla u_h \cdot \nabla v_h \, dx &\!-\! \int_{\mathcal{F}_{h}} (\llbracket u_h \rrbracket \cdot \{\nabla v_h\} \!+\! \llbracket v_h \rrbracket \cdot \{\nabla u_h\}) \, ds \!+\! \int_{\mathcal{F}_h} \frac{\sigma p^2}{h} \llbracket u_h \rrbracket \cdot \llbracket v_h \rrbracket \, ds \!=\! \!-\!\int_{\partial \Omega} g (\partial_{n} v_h) \, ds,\\
        & - \int_{\partial \mathcal{T}_h} (u_h \partial_n v_h + v_h \partial_n u_h) \, ds + \int_{\partial \mathcal{T}_h} \frac{\sigma p^2}{h} u_h v_h \, ds,
    \end{align}
    for all $v_h \in \mathbb{T}_h$, where $\mathcal{F}_h$ is the set of all faces in the mesh $\mathcal{T}_h$, $\sigma$ is a positive penalty parameter, and $h$ is the mesh size
\end{frame}
\begin{frame}
    \frametitle{An Eigenvalue Problem}
    $\newline$
    Since we have assembled the stiffness matrix, we can also assemble the mass matrix and consider the following eigenvalue problem: find $(\lambda_h, u_h) \in \mathbb{R} \times \mathbb{T}_h$ such that
    \begin{align}
        \int_{\mathcal{T}_h} \nabla u_h \cdot \nabla v_h \, dx &\!-\! \int_{\mathcal{F}_{h}} (\llbracket u_h \rrbracket \cdot \{\nabla v_h\} \!+\! \llbracket v_h \rrbracket \cdot \{\nabla u_h\}) \, ds \!+\! \int_{\mathcal{F}_h} \frac{\sigma p^2}{h} \llbracket u_h \rrbracket \cdot \llbracket v_h \rrbracket \, ds \\
        & - \int_{\partial \mathcal{T}_h} (u_h \partial_n v_h + v_h \partial_n u_h) \, ds + \int_{\partial \mathcal{T}_h} \frac{\sigma p^2}{h} u_h v_h \, ds = \lambda_h \int_{\mathcal{T}_h} u_h v_h \, dx,
    \end{align}
    for all $v_h \in \mathbb{T}_h$. The mass matrix is the standard DG mass matrix, i.e. $\int_{\mathcal{T}_h} u_h v_h \, dx$.
    \begin{itemize}
        \item[$\color{oxfordblue}\blacktriangleright$]<2-> The mass matrix $\underline{\underline{M}}$ only need to be the DG mass matrix, since conformity is already imposed in the stiffness matrix.
        \item[$\color{oxfordblue}\blacktriangleright$]<3-> The stiffness matrix is parameter dependent, i.e. $\underline{\underline{K}} = \underline{\underline{K}}_1 + \sigma \underline{\underline{K}}_2$.
    \end{itemize}
\end{frame}
\begin{frame}
    \frametitle{Parameter dependence}
    $\newline$
    \begin{itemize}
        \item [$\color{oxfordblue}\blacktriangleright$]<1-> The parameter $\sigma$ has to be chosen sufficiently large to ensure positive definiteness of the stiffness matrix $\underline{\underline{K}}$.
    \end{itemize}
\end{frame}
\end{document}

\usetikzlibrary{shapes,tikzmark,matrix}
\tikzset{every tikzmarknode/.style={%
        draw=red, semithick, inner sep=2pt}
        }
\usetikzlibrary{positioning, arrows.meta}
\usetikzlibrary{3d,perspective}
\tikzset
{
  axis/.style={thick,-latex},
  my view/.style={3d view={65}{20}},
  nutation/.style={rotate around x=\nut}
}
\usepackage{pgfplots}
\pgfplotsset{compat = newest}
\usetikzlibrary{intersections, pgfplots.fillbetween}
\ifpdf
  \DeclareGraphicsExtensions{.eps,.pdf,.png,.jpg}
\else
  \DeclareGraphicsExtensions{.eps}
\fi

\usepackage{fontawesome5}
\newcommand{\myframesubtitle}[1]{\framesubtitle{{\tiny \faChevronRight} #1}}


\newenvironment{refbox}
    { \begin{tcolorbox}[colframe=base1,colback=base3] 
        \begin{columns}
            \begin{column}[t]{0.01\textwidth}
                \centering
                \vspace{-0.4cm}
                { \faBookmark}
            \end{column}
            \begin{column}[t]{0.99\textwidth}
                \hspace{-0.4cm}
                \footnotesize}
                {
            \end{column}
        \end{columns}
    \end{tcolorbox}
    }
\usepackage{wasysym}


%%%%%%%%%%%%%%%%%%%%%%%%%%%%%%%%% MACROS %%%%%%%%%%%%%%%%%%%%%%%%%%%%%%%
\usepackage{mathtools}
\mathtoolsset{showonlyrefs}
\usepackage{todonotes}
\makeatletter
\newcommand{\mat}[1]{{\mathpalette\mat@{#1}}}
\newcommand{\mat@}[2]{%
  \begingroup
  \sbox\z@{$\m@th#1\underline{#2}$}%
  \dimen@=\dp\z@ \advance\dimen@ -2\mat@dimen{#1}%
  \dp\z@=\dimen@
  \sbox\z@{$\m@th\underline{\box\z@}$}%
  \box\z@
  \endgroup
}
\newcommand\mat@dimen[1]{%
  \fontdimen8
  \ifx#1\displaystyle\textfont\else
  \ifx#1\textstyle\textfont\else
  \ifx#1\scriptstyle\scriptfont\else
  \scriptscriptfont\fi\fi\fi 3
}
\newcommand{\tensor}[1]{{\mathpalette\tensor@{#1}}}
\newcommand{\tensor@}[2]{%
  \begingroup
  \sbox\z@{$\m@th#1\underline{#2}$}%
  \dimen@=\dp\z@ \advance\dimen@ -2\tensor@dimen{#1}%
  \dp\z@=\dimen@
  \sbox\z@{$\m@th\underline{\box\z@}$}%
  \box\z@
  \endgroup
}
\newcommand\tensor@dimen[1]{%
  \fontdimen8
  \ifx#1\displaystyle\textfont\else
  \ifx#1\textstyle\textfont\else
  \ifx#1\scriptstyle\scriptfont\else
  \scriptscriptfont\fi\fi\fi 3
}
\def\contraction{\vbox{\baselineskip2.5\p@ \lineskiplimit\z@
  \kern\p@\hbox{.}\hbox{.}\hbox{.}\hbox{.}}}
\makeatother
\newcommand{\duality}[4]{\prescript{}{#1}{\Big\langle} #2 , #3 \Big\rangle_{#4}}
\newcommand{\interp}{\newpage}
\newcommand{\abs}[1]{\lvert#1\rvert}
\newcommand{\norm}[1]{\lVert #1\rVert}
\newcommand{\cchevrons}[1]{\langle\!\langle #1 \rangle\!\rangle}
\newcommand{\ohat}[1]{\overset{\circ}{#1}}
\renewcommand{\v}{\vec{\text{v}}}
\DeclareMathOperator{\tr}{tr}
\let \vec \underline
\let \pt \mathbf

\usepackage{stmaryrd} 

% PRESENTATION INFO

\author[U. Zerbinati]{Umberto Zerbinati*} %ADD CO-AUTHORS IF NEEDED
\institute{*Mathematical Institute -- University of Oxford}

\title[Embedded Trefftz Maxwell]{
Embedded Trefftz Discretisations for the Maxwell
Eigenvalue Problem with Applications to MHD}
\date[Heidelberg, 1st Sep. '25]{ENUMATH, 1st September 2025} %[short for footer]{INSERT CONFERENCE NAME, full date for titlepage}


\begin{document}
% Setting colors for tcolorbox 
\tcbset{colframe=base02,colback=base2,coltext=base03,coltitle=base3}

\begin{frame}[plain,noframenumbering]
  \titlepage
\end{frame}

\begin{frame}
    \frametitle{Trefftz Methods}
    $\newline$
    The idea behind DG-Trefftz methods is to consider a discontinuous Galerkin method where the local approximation spaces are made of functions that are piecewise solutions of the target PDE.
    For example, let us consider the Laplace equation,
    \begin{equation}
        -\Delta u = 0 \quad \text{in } \Omega,  \quad u = g \quad \text{on } \partial \Omega.
    \end{equation} 
    \visible<2->{
        A DG-Trefftz method for this problem would consider a mesh $\mathcal{T}_h$ of $\Omega$ and a local discrete space
        \begin{equation}
            \mathbb{T}^p(K) = \{ v \in \mathbb{P}^p(K) : \Delta v = 0 \text{ in } K \}, \quad \forall K \in \mathcal{T}_h,
        \end{equation}
        where $\mathbb{P}^p(K)$ is the space of polynomials of degree at most $p$ on the element $K$.
    }
    \visible<3->{
        The global discrete space is then defined as
        \begin{equation}
            \mathbb{T}_h = \{ v_h \in L^2(\Omega) : v_h|_K \in \mathbb{T}^p(K), \forall K \in \mathcal{T}_h \}. 
        \end{equation}
    }
\end{frame}
\begin{frame}
    \textbf{No conformity} is imposed across element interfaces in the space $\mathbb{T}_h$, hence a DG formulation is needed to enforce the continuity of the solution.
    We thus consider the following DG formulation: find $u_h \in \mathbb{T}_h$ such that
    \begin{align}
        \int_{\mathcal{T}_h} \nabla u_h \cdot \nabla v_h \, dx &\!-\! \int_{\mathcal{F}_{h}} (\llbracket u_h \rrbracket \cdot \{\nabla v_h\} \!+\! \llbracket v_h \rrbracket \cdot \{\nabla u_h\}) \, ds \!+\! \int_{\mathcal{F}_h} \frac{\sigma p^2}{h} \llbracket u_h \rrbracket \cdot \llbracket v_h \rrbracket \, ds \!=\! \!-\!\int_{\partial \Omega} g (\partial_{n} v_h) \, ds,\\
        & - \int_{\partial \mathcal{T}_h} (u_h \partial_n v_h + v_h \partial_n u_h) \, ds + \int_{\partial \mathcal{T}_h} \frac{\sigma p^2}{h} u_h v_h \, ds,
    \end{align}
    for all $v_h \in \mathbb{T}_h$, where $\mathcal{F}_h$ is the set of all faces in the mesh $\mathcal{T}_h$, $\sigma$ is a positive penalty parameter, and $h$ is the mesh size
\end{frame}
\begin{frame}
    \frametitle{An Eigenvalue Problem}
    $\newline$
    Since we have assembled the stiffness matrix, we can also assemble the mass matrix and consider the following eigenvalue problem: find $(\lambda_h, u_h) \in \mathbb{R} \times \mathbb{T}_h$ such that
    \begin{align}
        \int_{\mathcal{T}_h} \nabla u_h \cdot \nabla v_h \, dx &\!-\! \int_{\mathcal{F}_{h}} (\llbracket u_h \rrbracket \cdot \{\nabla v_h\} \!+\! \llbracket v_h \rrbracket \cdot \{\nabla u_h\}) \, ds \!+\! \int_{\mathcal{F}_h} \frac{\sigma p^2}{h} \llbracket u_h \rrbracket \cdot \llbracket v_h \rrbracket \, ds \\
        & - \int_{\partial \mathcal{T}_h} (u_h \partial_n v_h + v_h \partial_n u_h) \, ds + \int_{\partial \mathcal{T}_h} \frac{\sigma p^2}{h} u_h v_h \, ds = \lambda_h \int_{\mathcal{T}_h} u_h v_h \, dx,
    \end{align}
    for all $v_h \in \mathbb{T}_h$. The mass matrix is the standard DG mass matrix, i.e. $\int_{\mathcal{T}_h} u_h v_h \, dx$.
    \begin{itemize}
        \item[$\color{oxfordblue}\blacktriangleright$]<2-> The mass matrix $\underline{\underline{M}}$ only need to be the DG mass matrix, since conformity is already imposed in the stiffness matrix.
        \item[$\color{oxfordblue}\blacktriangleright$]<3-> The stiffness matrix is parameter dependent, i.e. $\underline{\underline{K}} = \underline{\underline{K}}_1 + \sigma \underline{\underline{K}}_2$.
    \end{itemize}
\end{frame}
\begin{frame}
    \frametitle{Parameter dependence}
    $\newline$
    \begin{itemize}
        \item [$\color{oxfordblue}\blacktriangleright$]<1-> The parameter $\sigma$ has to be chosen sufficiently large to ensure positive definiteness of the stiffness matrix $\underline{\underline{K}}$.
    \end{itemize}
\end{frame}
\end{document}
